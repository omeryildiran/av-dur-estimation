
%% =============================================================================
%% RESULTS SECTION - Unimodal Duration Discrimination
%% Generated from weber_fractions_jnds_analysis.ipynb
%% =============================================================================

\section{Results}

\subsection{Unimodal Duration Discrimination Precision}

To characterize temporal discrimination precision across modalities and 
reliability conditions, we fitted cumulative Gaussian psychometric functions 
to the 2IFC duration discrimination data. Each psychometric function was 
parameterized by a lapse rate ($\lambda$), a bias parameter ($\mu$), and a 
discrimination threshold ($\sigma$). Lower $\sigma$ values indicate better 
temporal precision.

\subsubsection{Bootstrap Analysis on Pooled Data}

Psychometric functions were first fitted to pooled data across all participants 
using maximum likelihood estimation. Bootstrap resampling ($N = 5$ 
iterations, resampling participants with replacement) was used to estimate 
95\% confidence intervals for the discrimination threshold $\sigma$.

Results revealed a clear ordering of temporal precision across conditions 
(Table~\ref{tab:precision}). The \textit{low-noise} auditory condition 
showed the best discrimination precision ($\sigma = 140$~ms, 
95\% CI $[127, 152]$), 
corresponding to a Weber fraction of 0.27. The visual condition 
showed intermediate precision ($\sigma = 291$~ms, 95\% CI 
$[244, 351]$; Weber fraction 
= 0.68). The \textit{high-noise} auditory condition showed the 
worst precision ($\sigma = 367$~ms, 95\% CI 
$[291, 457]$; Weber fraction 
= 1.74).

Bootstrap comparisons confirmed that precision in the high-noise auditory 
condition was significantly worse than in the low-noise condition, with 
non-overlapping 95\% confidence intervals ($P(\sigma_{\text{high}} > 
\sigma_{\text{low}}) = 1.000$). This 2.6-fold 
difference in discrimination threshold demonstrates the effectiveness of our 
auditory reliability manipulation.

\subsubsection{Individual Participant Analysis}

To complement the bootstrap analysis and assess effect consistency across 
participants, we fitted psychometric functions separately for each participant 
($N = 11$) in each condition. This approach weights each 
participant equally, regardless of the number of trials completed.

A Friedman test revealed a significant overall difference in $\sigma$ across 
the three conditions ($\chi^2(2) = 18.73$, $p < .001$). 
Follow-up pairwise comparisons were conducted using Wilcoxon signed-rank tests 
with Bonferroni correction ($\alpha = .0167$).

The auditory reliability manipulation produced a robust effect at the 
individual participant level: discrimination thresholds were significantly 
lower in the low-noise condition (median $= 123$~ms, 
$M = 134$~ms, $SD = 51$~ms) compared 
to the high-noise condition (median $= 939$~ms, 
$M = 868$~ms, $SD = 565$~ms; 
$W = 0$, $p = 0.0010$, $r = 1.00$). The large effect size 
($r = 1.00$) indicates that 11 of 11 participants 
showed the expected pattern of worse precision under high-noise conditions.

Visual duration discrimination (median $\sigma = 263$~ms, 
$M = 341$~ms, $SD = 157$~ms) was 
significantly worse than the low-noise auditory condition ($W = 0$, 
$p = 0.0010$, $r = 1.00$). The comparison between visual and high-noise 
auditory conditions did not reach significance after Bonferroni correction 
($W = 7$, $p = 0.0186$), suggesting that visual precision was 
intermediate between the two auditory conditions.

\subsubsection{Just Noticeable Differences}

The 75\% just noticeable differences (JNDs) were calculated as 
$\text{JND} = 0.6745 \times \sigma \times T_s$, where $T_s = 500$~ms 
is the standard duration. Based on the individual participant fits, the JNDs 
were 90~ms for the low-noise auditory condition, 
230~ms for the visual condition, and 586~ms for 
the high-noise auditory condition.

%% =============================================================================
%% TABLES
%% =============================================================================

\begin{table}[htbp]
\centering
\caption{Temporal discrimination precision ($\sigma$) across experimental 
conditions. Bootstrap estimates are based on pooled data with participant-level 
resampling ($N = 5$ iterations). Individual participant statistics are 
computed across $N = 11$ participants. Weber fractions are 
calculated as $\sigma / T_s$ where $T_s = 500$~ms.}
\label{tab:precision}
\begin{tabular}{lcccccc}
\hline
 & \multicolumn{2}{c}{Bootstrap (Pooled)} & \multicolumn{3}{c}{Individual Fits} & \\
\cline{2-3} \cline{4-6}
Condition & $\sigma$ (ms) & 95\% CI & $M$ (ms) & $SD$ & Median & Weber \\
\hline
Auditory Low-Noise  & 140 & [127, 152] & 134 & 51 & 123 & 0.27 \\
Visual              & 291 & [244, 351] & 341 & 157 & 263 & 0.68 \\
Auditory High-Noise & 367 & [291, 457] & 868 & 565 & 939 & 1.74 \\
\hline
\end{tabular}
\end{table}

\begin{table}[htbp]
\centering
\caption{Pairwise statistical comparisons of temporal discrimination precision. 
Wilcoxon signed-rank tests were used with Bonferroni correction 
($\alpha = .0167$). Effect sizes are reported as rank-biserial correlations 
($r$). Bootstrap $P$ indicates the proportion of bootstrap samples where the 
first condition showed higher $\sigma$ than the second.}
\label{tab:comparisons}
\begin{tabular}{lccccc}
\hline
Comparison & $W$ & $p$ & $r$ & Bootstrap $P$ & Sig. \\
\hline
Aud High-Noise vs Low-Noise & 0 & 0.0010 & 1.00 & 1.000 & Yes \\
Visual vs Aud Low-Noise     & 0 & 0.0010 & 1.00 & 1.000 & Yes \\
Visual vs Aud High-Noise    & 7 & 0.0186 & 0.79 & 0.400 & No$^\dagger$ \\
\hline
\multicolumn{6}{l}{$^\dagger$Not significant after Bonferroni correction}
\end{tabular}
\end{table}

%% =============================================================================
%% SUMMARY STATISTICS FOR INLINE REPORTING
%% =============================================================================

%% Bootstrap results:
%% - Auditory Low-Noise: sigma = 140 ms [95% CI: 127-152]
%% - Visual: sigma = 291 ms [95% CI: 244-351]
%% - Auditory High-Noise: sigma = 367 ms [95% CI: 291-457]
%% - Reliability effect ratio: 2.6x

%% Individual participant results:
%% - Auditory Low-Noise: M = 134 ms, SD = 51, Median = 123
%% - Visual: M = 341 ms, SD = 157, Median = 263
%% - Auditory High-Noise: M = 868 ms, SD = 565, Median = 939
%% - Reliability effect ratio: 6.5x
%% - Effect consistency: 11/11 participants showed expected effect

%% Statistical tests:
%% - Friedman: chi2(2) = 18.73, p = 0.0001
%% - Aud High vs Low: W = 0, p = 0.0010, r = 1.00
%% - Vis vs Aud Low: W = 0, p = 0.0010, r = 1.00
%% - Vis vs Aud High: W = 7, p = 0.0186, r = 0.79

%% Weber fractions:
%% - Auditory Low-Noise: 0.267
%% - Visual: 0.682
%% - Auditory High-Noise: 1.737

%% JNDs (75%):
%% - Auditory Low-Noise: 90 ms
%% - Visual: 230 ms
%% - Auditory High-Noise: 586 ms
